%%%%%%%%%%%%%%%%%%%%%%%%%%%%%%%%%%%%%%%%%%%%%%%%%%%%%%%%%%%%%%%%%%%%%%
% How to use writeLaTeX: 
%
% You edit the source code here on the left, and the preview on the
% right shows you the result within a few seconds.
%
% Bookmark this page and share the URL with your co-authors. They can
% edit at the same time!
%
% You can upload figures, bibliographies, custom classes and
% styles using the files menu.
%
%%%%%%%%%%%%%%%%%%%%%%%%%%%%%%%%%%%%%%%%%%%%%%%%%%%%%%%%%%%%%%%%%%%%%%

\documentclass[15pt]{article}

\usepackage{sbc-template}

\usepackage{graphicx,url}

%\usepackage[brazil]{babel}   
\usepackage[utf8]{inputenc}  

     
\sloppy

\title{Sistemas Distribuídos para Redes Móveis \\ Visando Veículos Aéreos Não Tripulados (VANT)}

\author{Guilherme Cosso Lima Pimenta\inst{1} \and Fátima de L. P. D. Figueiredo\inst{2}}


\address{%
  Instituto de Ciências Exatas e Informática\\
  Pontifícia Universidade Católica de Minas Gerais (PUC Minas)\\
  Caixa Postal 1.686  30.535-901 - Belo Horizonte, MG - Brazil
}

\begin{document} 

\maketitle

\begin{abstract}
\par The presence of Unmanned Aerial Vehicles (UAVs) in the current scenario has grown significantly. Nowadays, they play an essential role in events in both rural and urban areas, with a direct impact on the global community. This work aims to conduct simulations in the SMPL (Process Modeling and Simulation System) of Unmanned Aerial Vehicles, in which communication among these aircraft is established through a distributed control system, aimed at improving fault management in their operations

    
\end{abstract} 



\begin{resumo} 

\par A presença de Veículos Aéreos Não Tripulados (VANTs) no cenário atual cresceu significativamente. Atualmente, eles desempenham um papel essencial em eventos tanto em áreas rurais como urbanas, e têm um impacto direto na comunidade global. Este trabalho tem como objetivo realizar simulações no SMPL (Sistema de Modelagem e Simulação de Processos) de Veículos Aéreos Não Tripulados, nos quais a comunicação entre essas aeronaves seja estabelecida por meio de um sistema distribuído de controle, que visa aprimorar o gerenciamento de falhas em suas operações.

\end{resumo}

\section{Introdução}
Neste artigo, abordamos o tema...



\end{document}
